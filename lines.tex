% Spacing lines
\newdimen\leading

\def\linespacing#1{\baselineskip=#1em}
\def\nointerlineskip{\prevdepth=-1000pt}
\def\offinterlineskip{\baselineskip=-1000pt\lineskip=0pt\lineskiplimit=\maxdimen}
\def\titleleading{\linespacing1}

% Skipping lines
\def\skipline{\vskip\leading plus 0pt minus 0pt}
\def\skiphalfline{\vskip.5\leading plus 0pt minus 0pt}

% Making single line boxes
\def\line{\hbox to\hsize}
\def\leftline#1{\line{#1\hss}}
\def\centerline#1{\line{\hss#1\hss}}
\def\rightline#1{\line{\hss#1}}

% TODO: Define this elsewhere.
\leading=15pt
\baselineskip=\leading % desired glue between baselines
\lineskip=0pt % interline glue if \baselineskip isn't feasible
\lineskiplimit=0pt % threshold where \baselineskip changes to \lineskip
\leftskip=0pt % glue at left of justified lines
\rightskip=0pt % glue at right of justified lines

\pretolerance=100 % badness tolerance before hyphenation
\tolerance=200 % badness tolerance after hyphenation
\linepenalty=10 % amount added to badness of every line in a paragraph
\adjdemerits=10000 % demerits for adjacent incompatible lines
\hyphenpenalty=10 % penalty for line break after discretionary hyphen
\exhyphenpenalty=50 % penalty for like break after explicit hyphen
\binoppenalty=700 % penalty for line break after binary operation
\relpenalty=500 % penalty for line break after math relation
\doublehyphendemerits=10000 % demerits for consecutive broken lines
\finalhyphendemerits=5000 % demerits for a penultimate broken line
\lefthyphenmin=3 % smallest fragment at beginning of hyphenated word
\righthyphenmin=3 % smallest fragment at end of hyphenated word
