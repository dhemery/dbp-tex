% Make this format with:
% pdftex -ini -enc dbp.tex
\catcode`\{=1 % { begin group
\catcode`\}=2 % } end group
\catcode`\$=3 % $ shift into and out of math mode
\catcode`\&=4 % & alignment tab
\catcode`\#=6 % # macro parameter prefix
\catcode`\^=7 % ^ superscript operator
\catcode`\_=8 % _ subscript operator
\catcode`\^^I=10 % Treat tab as like space
\catcode`\~=13 % Tilde is active
\catcode`\^^L=13 \outer\def\^^L{\par}   % Convert form feed to \par

% utf8-t1 requires \bgroup and \loop
\let\bgroup={ \let\egroup=}
\def\loop#1\repeat{\def\body{#1}\iterate}
\def\iterate{\body \let\next\iterate \else\let\next\relax\fi \next}
\let\repeat=\fi

% Load the mapping from UTF-8 input to T1 output.
\input utf8-t1

\countdef\nextboxregister=20 \nextboxregister=10
\countdef\nextcountregister=21 \nextcountregister=30
\countdef\nextdimenregister=22 \nextdimenregister=10
\countdef\nextskipregister=23 \nextskipregister=10
\countdef\nexttoksregister=24 \nexttoksregister=10

\def\newcount#1{\global\countdef#1=\nextcountregister \advance\nextcountregister by 1}
\def\newdimen#1{\global\dimendef#1=\nextdimenregister \advance\nextdimenregister by 1}
\def\newskip#1{\global\skipdef#1=\nextskipregister \advance\nextskipregister by 1}
\def\newtoks#1{\global\toksdef#1=\nexttoksregister \advance\nexttoksregister by 1}

\def\optionalwithdefault#1#2{\if\relax#1\relax#2\else#1\fi}
\def\optionalwithprefix#1#2{\if\relax#2\relax\else#1#2\fi}
\def\fontfilename{\fontbasename\optionalwithprefix{-}{\fontshape}-\fontfigures\optionalwithprefix{-}{\fontvariant}-\fontencoding\optionalwithprefix{ at }{\fontsize\relax}}
\def\fontshape{\optionalwithdefault{\fontweight\fontslant\fontopticalsize}{\fontdefaultshape}}
\def\fonttag#1{\fontstyle#1font}

% Declare a new font family.
% This specifies terms to identify parts of file names for this family
\def\newfamily#1#2#3#4#5#6{% #1 name for this family
  \expandafter\xdef\csname#1basename\endcsname{#2}% #2: the words that identify the family in this family's file names (e.g. GaramondPremrPro)
  \expandafter\xdef\csname#1regularweight\endcsname{#3}% #3: the possibly empty word that identifies the regular weight in this family's file names
  \expandafter\xdef\csname#1italicslant\endcsname{#4}% #4: the word that identifies italic slant in this family's file names
  \expandafter\xdef\csname#1boldweight\endcsname{#5}% #5: the word that identifies bold weight in this family's file names
  \expandafter\xdef\csname#1defaultshape\endcsname{#6}% #6: the possibly empty word that identifies regular roman book style in this family's file names
}

% Declare a new font style. This specifies the terms to use
% to build file names for this style.
\def\newstyle#1#2#3#4#5{% #1: a name for the style
  \expandafter\xdef\csname#1family\endcsname{#2}% #2: the name of this style's font family
  \expandafter\xdef\csname#1opticalsize\endcsname{#3}% #3: the word that identifies this style's optical size in the family's file names
  \expandafter\xdef\csname#1figures\endcsname{#4}% #4: the word that identifies this style's figure style in the family's file names
  \expandafter\xdef\csname#1size\endcsname{#5}% #5: the possibly empty point size at which to render this style
}

% Select a style and switch to its \rm font
\def\style#1{%
  \edef\fontstyle{#1}%
  \edef\fontfamily{\csname#1family\endcsname}%
  \edef\fontbasename{\csname\fontfamily basename\endcsname}%
  \edef\fontdefaultshape{\csname\fontfamily defaultshape\endcsname}%
  \edef\fontopticalsize{\csname#1opticalsize\endcsname}%
  \edef\fontfigures{\csname#1figures\endcsname}%
  \edef\fontsize{\csname#1size\endcsname}%
  \rm%
}

\def\withfontbasename#1{\edef\fontbasename{#1}}
\def\withfontencoding#1{\edef\fontencoding{#1}}
\def\withfontfigures#1{\edef\fontfigures{#1}}
\def\withfontopticalsize#1{\edef\fontopticalsize{#1}}
\def\withfontshape#1#2#3{\withfontweight{#1}\withfontslant{#2}\withfontopticalsize{#3}}
\def\withfontslant#1{\edef\fontslant{#1}}
\def\withfontvariant#1{\edef\fontvariant{#1}}
\def\withfontweight#1{\edef\fontweight{#1}}

\def\loadfont#1{%
  \ifcsname#1\endcsname\else%
    \global\expandafter\font\csname#1\endcsname=\fontfilename%
    \makeexpandable{\csname#1\endcsname}%
  \fi%
}

\def\selectfont#1{%
  \edef\selectedfonttag{\fonttag{#1}}%
  \loadfont{\selectedfonttag}%
  \csname\selectedfonttag\endcsname%
}

\def\rm{\withfontweight{\csname\fontfamily regularweight\endcsname}\withfontslant{}\withfontvariant{}\withfontencoding{t1}\selectfont{rm}}
\def\it{\withfontslant{\csname\fontfamily italicslant\endcsname}\selectfont{it}}
\def\bf{\withfontweight{\csname\fontfamily boldweight\endcsname}\selectfont{bf}}
\def\sc{\withfontvariant{sc}\selectfont{sc}}
\def\tl{\withfontvariant{titling}\withfontfigures{tlf}\selectfont{tl}}
\def\sym{\withfontencoding{ts1}\selectfont{sym}}

%%%%%%%%%%%%%%%%%%%%%%%%%%%%%%%%%%%%%%%%%%%%%%%%%%%%%%%%%%%
%
% Text and Title Styles
%
%%%%%%%%%%%%%%%%%%%%%%%%%%%%%%%%%%%%%%%%%%%%%%%%%%%%%%%%%%%

\def\aheadsize{36pt} % Book and part titles
\def\bheadsize{24pt} % Author name
\def\cheadsize{18pt} % Chapter titles and publisher logo
\def\bodytextsize{11pt}
\def\smalltextsize{10pt}
\def\runningheadersize{8pt}

\def\definestyles{%
  \newstyle{booktitle}{\booktitlefamily}{\aheadopticalsize}{tlf}{\aheadsize}%
  \newstyle{authorname}{\authornamefamily}{\bheadopticalsize}{tlf}{\bheadsize}%
  \newstyle{parttitle}{\parttitlefamily}{\aheadopticalsize}{osf}{\aheadsize}%
  \newstyle{chaptertitle}{\chaptertitlefamily}{\cheadopticalsize}{osf}{\cheadsize}%
  \newstyle{bodytext}{\bodytextfamily}{\bodytextopticalsize}{osf}{\bodytextsize}%
  \newstyle{smalltext}{\smalltextfamily}{\smalltextopticalsize}{osf}{\smalltextsize}%
  \newstyle{runningheader}{\runningheaderfamily}{\runningheaderopticalsize}{osf}{\runningheadersize}%
}

%%%%%%%%%%%%%%%%%%%%%%%%%%%%%%%%%%%%%%%%%%%%%%%%%%%%%%%%%%%
%
% Textblock Size and Position on the Page
%
%%%%%%%%%%%%%%%%%%%%%%%%%%%%%%%%%%%%%%%%%%%%%%%%%%%%%%%%%%%

\newdimen\edgemargin
\newdimen\spinemargin
\def\setspinemargin#1{
  \edgemargin=#1
  \spinemargin=\pdfpagewidth
    \advance\spinemargin by -\hsize
    \advance\spinemargin by -#1
}

% Use edge margin for left/even pages,
% spine margin for odd/right pages
\def\leftmargin{\ifodd\pageno\edgemargin\else\spinemargin\fi}

%%%%%%%%%%%%%%%%%%%%%%%%%%%%%%%%%%%%%%%%%%%%%%%%%%%%%%%%%%%
%
% Output Routine
%
%%%%%%%%%%%%%%%%%%%%%%%%%%%%%%%%%%%%%%%%%%%%%%%%%%%%%%%%%%%

\newtoks\headline
\newtoks\footline
\newdimen\headlineskip % Distance from textblock first baseline to top of headline
\newdimen\footlineskip % Distance from textblock last baseline to footline baseline

% Write one page (headline, textblock, and footline).
\def\facingpages{%
  \shipout\vbox{\moveright\leftmargin\vbox{\makeheadline\pagebody\makefootline}}%
  \userunningpageframe%
  \advancepageno%
}

\def\pagebody{\vbox to\vsize{\unvbox255}}

\def\makeheadline{\vbox to 0pt{\positionheadline\line{\the\headline}\vss\nointerlineskip}}
\def\makefootline{\positionfootline\line{\the\footline}}
\def\positionheadline{\vskip-\headlineposition}
\def\positionfootline{\baselineskip=\footlineposition \lineskiplimit=0pt}

%%%%%%%%%%%%%%%%%%%%%%%%%%%%%%%%%%%%%%%%%%%%%%%%%%%%%%%%%%%
%
% Page Frame (Headline and Footline)
%
%%%%%%%%%%%%%%%%%%%%%%%%%%%%%%%%%%%%%%%%%%%%%%%%%%%%%%%%%%%

\def\runningheadline{\line{}}
\def\runningfootline{\line{}}
\def\useclearedframethispage{\global\headline={\clearedheadline}\global\footline={\clearedfootline}}
\def\usedisplayframethispage{\global\headline={\displayheadline}\global\footline={\displayfootline}}
\def\userunningpageframe{\global\headline={\runningheadline}\global\footline={\runningfootline}}

% Running headline and footline for cleared pages
% (pages begun by \nextpage, \nextoddpage, and \nextevenpage).
% You may want to change these for debugging.
\def\clearedheadline{\line{}}
\def\clearedfootline{\line{}}

% Running headline and footline for display pages
% (pages that begin with a major display item such as a part or chapter title)
\def\displayheadline{\line{}}
\def\displayfootline{\style{runningheader}\sc\hfil\folio\hfil}

% Running headline and footline after user calls \useemptypageframe.
% By default, this format uses empty frames for frontmatter and backmatter pages.
% You may want to change these for debugging.
\def\emptyheadline{\line{}}
\def\emptyfootline{\line{}}

% Running headline and footline for most content pages.
% We use different headlines on even and odd pages.
% Redefine to suit your formatting needs.
\def\normalheadline{\style{runningheader}\sc\ifodd\pageno\normaloddpageheadline\else\normalevenpageheadline\fi}
% No footline on normal pages
\def\normalfootline{\line{}}
% Odd (right) pages: title at inner margin, page number at outer margin
\def\normaloddpageheadline{\style{runningheader}\sc\title\hfil\folio}
% Even (left) pages: page number at outer margin, author name at inner margin
\def\normalevenpageheadline{\style{runningheader}\sc\folio\hfil\author}

\def\useemptypageframe{
  \gdef\runningheadline{\emptyheadline}
  \gdef\runningfootline{\emptyfootline}
  \userunningpageframe
}

\def\usenormalpageframe{
  \gdef\runningheadline{\normalheadline}
  \gdef\runningfootline{\normalfootline}
  \userunningpageframe
}

%%%%%%%%%%%%%%%%%%%%%%%%%%%%%%%%%%%%%%%%%%%%%%%%%%%%%%%%%%%
%
% Pages
%
%%%%%%%%%%%%%%%%%%%%%%%%%%%%%%%%%%%%%%%%%%%%%%%%%%%%%%%%%%%

\def\nextpage{\vfil\eject\useclearedframethispage}
\def\nextoddpage{\nextpage\onoddpage}
\def\nextevenpage{\nextpage\ifodd\pageno \line{}\nextpage\else \fi}
\def\onoddpage{\ifodd\pageno \else \line{}\nextpage\fi}

% Breaking pages
\def\break{\penalty-10000 }
\def\nobreak{\penalty10000 }
\def\eject{\par\break}
\def\raggedbottom{}

% Page numbering
\newcount\pagenoincrement
\newcount\pageno

\def\advancepageno{\global\advance\pageno by \pagenoincrement}
\def\folio{\number\pageno}

%%%%%%%%%%%%%%%%%%%%%%%%%%%%%%%%%%%%%%%%%%%%%%%%%%%%%%%%%%%
%
% Divisions
%
%%%%%%%%%%%%%%%%%%%%%%%%%%%%%%%%%%%%%%%%%%%%%%%%%%%%%%%%%%%

\def\frontmatter{\nextoddpage\global\pageno=-1\global\pagenoincrement=-1\useemptypageframe}
\def\mainmatter{\nextoddpage\global\pageno=1\global\pagenoincrement=1\usenormalpageframe}
\def\backmatter{\frontmatter}
\def\endbook{\nextoddpage\end}

\def\part#1{
  \nextoddpage
  \baselinebox{\raggedspine\parttitle{#1}\par}{10}{0}
  \nextoddpage
}

\def\chapter#1{
  \nextpage
  \usedisplayframethispage
  \baselinebox{\raggedspine\chaptertitle{#1}}{10}{1}
}

% \bookinfo is a named frontmatter or backmatter division.
\def\bookinfo#1{
  \nextpage
  \useemptypageframe
  {\centered\chaptertitle{#1}\par}
}

%%%%%%%%%%%%%%%%%%%%%%%%%%%%%%%%%%%%%%%%%%%%%%%%%%%%%%%%%%%
%
% Titles
%
%%%%%%%%%%%%%%%%%%%%%%%%%%%%%%%%%%%%%%%%%%%%%%%%%%%%%%%%%%%

% \baselinebox{content}{bottom (baselines)}{skip (baselines)}
\long\def\baselinebox#1#2#3{\vbox to #2\leading{\line{}\vfill#1}\vskip#3\leading}

\def\booktitle#1{{\style{booktitle}\titleleading\uppercase\expandafter{#1}\par}}
\def\authorname#1{{\style{authorname}\titleleading\uppercase\expandafter{#1}\par}}
\def\parttitle#1{{\style{parttitle}\sc\titleleading#1\par}}
\def\chaptertitle#1{{\style{chaptertitle}\sc\titleleading#1\par}}

%%%%%%%%%%%%%%%%%%%%%%%%%%%%%%%%%%%%%%%%%%%%%%%%%%%%%%%%%%%
%
% Logo
%
%%%%%%%%%%%%%%%%%%%%%%%%%%%%%%%%%%%%%%%%%%%%%%%%%%%%%%%%%%%
\newdimen\logotextsize \logotextsize=\cheadsize
\newfamily{proxima}{ProximaNova}{Regular}{It}{Bold}{Regular}
\newstyle{logo}{proxima}{}{osf}{\the\logotextsize}

\newdimen\logolinespacing
\newdimen\logobarthickness

\def\logobar{\hrule height\logobarthickness}
\def\logolineskip{\vskip\logolinespacing}

\def\logostack#1#2{%
  \halign{##&##&##\cr%
    &DRISCOLL&\cr%
    &#1BROOK#2&\cr%
    &#1PRESS#2&\cr%
  }%
}

\def\logo#1{
  \begingroup%
  \style{logo}%
  \offinterlineskip%
  \lineskip=\logolinespacing%
  \vbox{%
    \logobar%
    \logolineskip%
    #1%
    \logolineskip%
    \logobar%
    }%
  \endgroup%
}

\def\leftlogo{\logo{\logostack{}{\hss}}}
\def\centerlogo{\logo{\logostack{\hss}{\hss}}}
\def\rightlogo{\logo{\logostack{\hss}{}}}
\def\linelogo{\logo{\hbox{DRISCOLL BROOK PRESS}}}

\logolinespacing=.25\logotextsize
\logobarthickness=0.25\logolinespacing

%%%%%%%%%%%%%%%%%%%%%%%%%%%%%%%%%%%%%%%%%%%%%%%%%%%%%%%%%%%
%
% Paragraphs
%
%%%%%%%%%%%%%%%%%%%%%%%%%%%%%%%%%%%%%%%%%%%%%%%%%%%%%%%%%%%

\def\raggedleft{\parindent=0pt\parfillskip=0pt\leftskip=0pt plus \hsize\relax}
\def\raggedright{\parindent=0pt\parfillskip=0pt\rightskip=0pt plus \hsize\relax}
\def\centered{\raggedleft\raggedright}
\def\raggedspine{\ifodd\pageno\raggedleft\else\raggedright\fi}

\def\authorinfoheading#1{\skipline\skipline{\sc\centered#1\par}}
\def\authorinfoitem#1{\skiphalfline{\centered#1\par}}
\def\connect#1#2{\authorinfoheading{#1}\authorinfoitem{#2}}
\def\genre#1{\authorinfoheading{#1}}
\def\genrebook#1{\authorinfoitem{\emph{#1}}}

\def\leadin#1{\nobreak\noindent{\sc #1}}

%%%%%%%%%%%%%%%%%%%%%%%%%%%%%%%%%%%%%%%%%%%%%%%%%%%%%%%%%%%
%
% Lines
%
%%%%%%%%%%%%%%%%%%%%%%%%%%%%%%%%%%%%%%%%%%%%%%%%%%%%%%%%%%%

% Spacing lines
\newdimen\leading

\def\linespacing#1{\baselineskip=#1em}
\def\normalbaselines{ % TODO: Delete this?
  \lineskip=\normallineskip
  \baselineskip=\normlbaselineskip
  \lineskiplimit=\normallineskiplimit
}
\def\nointerlineskip{\prevdepth=-1000pt} % TODO: What is this?
\def\offinterlineskip{ % TODO: What is this?
  \baselineskip=-1000pt
  \lineskip=0pt
  \lineskiplimit=\maxdimen
}
\def\titleleading{\linespacing1}

% Skipping lines
\def\skipline{\vskip\leading plus 0pt minus 0pt}
\def\skiphalfline{\vskip.5\leading plus 0pt minus 0pt}

% Making single line boxes
\def\line{\hbox to\hsize}
\def\leftline#1{\line{#1\hss}}
\def\centerline#1{\line{\hss#1\hss}}
\def\rightline#1{\line{\hss#1}}

%%%%%%%%%%%%%%%%%%%%%%%%%%%%%%%%%%%%%%%%%%%%%%%%%%%%%%%%%%%
%
% Characters
%
%%%%%%%%%%%%%%%%%%%%%%%%%%%%%%%%%%%%%%%%%%%%%%%%%%%%%%%%%%%

\def\'#1{{\accent"13 #1}}
\chardef\$=`\$
\def\copyright{{\sym\char"A9}}
\def\registered{{\sym\char"AB}}
\def\trademark{{\sym\char"97}}
\def\ldots{.\thinnbsp.\thinnbsp.}

% Character Styles
% TODO: Make this handle nested emphasis.
%       Or give a way to go back to roman.
%       Or switch to {\rm ...} and {\it ...}.
\def\emph#1{{\it #1\/}}

%%%%%%%%%%%%%%%%%%%%%%%%%%%%%%%%%%%%%%%%%%%%%%%%%%%%%%%%%%%
%
% Spaces
%
%%%%%%%%%%%%%%%%%%%%%%%%%%%%%%%%%%%%%%%%%%%%%%%%%%%%%%%%%%%
\def\nbsp{\kern1em }
\newdimen\enwidth   \def\enspace{\hskip\enwidth\relax}     \def\ennbsp{\kern\enwidth }
\newdimen\thinwidth \def\thinspace{\hskip\thinwidth\relax} \def\thinnbsp{\kern\thinwidth }
\newdimen\hairwidth \def\hairspace{\hskip\hairwidth\relax} \def\hairnbsp{\kern\hairwidth }

%%%%%%%%%%%%%%%%%%%%%%%%%%%%%%%%%%%%%%%%%%%%%%%%%%%%%%%%%%%
%
% Microtypography
%
%%%%%%%%%%%%%%%%%%%%%%%%%%%%%%%%%%%%%%%%%%%%%%%%%%%%%%%%%%%

\def\makeexpandable#1{\pdffontexpand#1 30 30 10 autoexpand}

%%%%%%%%%%%%%%%%%%%%%%%%%%%%%%%%%%%%%%%%%%%%%%%%%%%%%%%%%%%
%
% Default Values
%
%%%%%%%%%%%%%%%%%%%%%%%%%%%%%%%%%%%%%%%%%%%%%%%%%%%%%%%%%%%

\newdimen\maxdimen \maxdimen=16383.99999pt

% PDF settings. Adapted from pdftexconfig.tex
\pdfoutput=1
\pdfcompresslevel=9
\pdfdecimaldigits=3
\pdfhorigin=0 true in
\pdfvorigin=0 true in
\pdfpkresolution=600
\pdfminorversion=5
\pdfobjcompresslevel=2

% Paper
\pdfpagewidth=5 true in
\pdfpageheight=8 true in

% Page Output
\output{\facingpages}
\pageno=1
\pagenoincrement=1
% \tracingoutput % positive if showing boxes that are shipped out

% Textblock
\hsize=22pc % textblock width
\vsize=39pc % textblock height page height
\voffset=4.5pc % textblock top margin
\setspinemargin{4.5pc} % textblock spine margin
\topskip=10pt % glue at top of main pages

% Page Frames
\def\headlineposition{2\leading} % Place headline two lines above first baseline
\def\footlineposition{2\leading} % Place footline two lines below the textblock's last baseline

% Page Breaks
% \interlinepenalty % penalty for page break between lines
% \outputpenalty % penalty at the current page break
% \tracingpages % positive if showing page-break calculations
\raggedbottom
\brokenpenalty=100 % penalty for page break after a hyphenated line
\predisplaypenalty=10000 % penalty for page break just before a display
\clubpenalty=150 % penalty for create a club line at bottom of page
\widowpenalty=150 % penalty for creating a widow line at top of page
\displaywidowpenalty=50 % ditto, before a display

% Lines
\leading=15pt
\baselineskip=\leading % desired glue between baselines
% \lineskip % interline glue if \baselineskip isn't feasible
% \lineskiplimit % threshold where \baselineskip changes to \lineskip
% \leftskip % glue at left of justified lines
% \rightskip % glue at right of justified lines
% \spaceskip % glue between words, if nonzero
% \xspaceskip, % glue between sentences, if nonzero

% Hyphenation
\defaulthyphenchar=`\- % \hyphencharacter value when a font is loaded
\uchyph=1 % positive if hyphenating words beginning with capital letters

% Line Breaks
\pretolerance=100 % badness tolerance before hyphenation
\tolerance=200 % badness tolerance after hyphenation
\linepenalty=10 % amount added to badness of every line in a paragraph
\adjdemerits=10000 % demerits for adjacent incompatible lines
\hyphenpenalty=50 % penalty for line break after discretionary hyphen
\exhyphenpenalty=50 % penalty for like break after explicit hyphen
\binoppenalty=700 % penalty for line break after binary operation
\relpenalty=500 % penalty for line break after math relation
\doublehyphendemerits=10000 % demerits for consecutive broken lines
\finalhyphendemerits=5000 % demerits for a penultimate broken line
\lefthyphenmin=3 % smallest fragment at beginning of hyphenated word
\righthyphenmin=3 % smallest fragment at end of hyphenated word
% \language % the current set of hyphenation rules

% Microtypography
\pdfadjustspacing=2
\pdfprotrudechars=2
\enwidth=.5em
\thinwidth=.2em
\hairwidth=.1em

% Paragraphs
\parskip=0pt plus 1pt % extra glue just above paragraphs
\parindent=\leading % width of \indent
\emergencystretch=1.5\leading % reduces badness on final pass of line-breaking
\parfillskip=0pt plus 1fil % additional rightskip at end of paragraphs
% \looseness % change to the number of lines in a paragraph
% \tracingparagraphs % positive if showing line-break calculations
% \hangindent % amount of hanging indentation
% \hangafter % hanging indentation changes after this many lines

% Debugging
\errorcontextlines=5 % maximum extra context shown when errors occur
\hbadness=1000 % badness above which bad hboxes will be shown
\vbadness=1000 % badness above which bad vboxes will be shown
\hfuzz=0.1pt % maximum overrun before overfull hbox messages occur
\vfuzz=0.1pt % maximum overrun before overfull vbox messages occur
\overfullrule=5pt % width of rules appended to overfull boxes
\showboxbreadth=5 % maximum items per level when boxes are shown
\showboxdepth=3 % maximum level when boxes are shown
\tracinglostchars=1 % positive if showing characters not in the font
% \tracingonline % positive if showing diagnostic info on the terminal
% \tracingmacros % positive if showing macros as they are expanded
% \tracintstats % positive if showing statistics about memory usage
% \tracingcommands % positive if showing commands before they are executed
% \tracingrestores % positive if showing deassinments when groups end

% Integer parameters (TeXbook page 272)
\delimiterfactor=901 % ratio for variable delimiters, times 1000
\defaultskewchar=-1 % \skewchar value when a font is loaded
\newlinechar=-1 % character that starts a new output line
% \floatingpenalty % penalty for insertions that are split
% \pausing % positive if pausing after each line is read from a file
% \holdinginserts % positive if insertions remain dormant in output box
% \globaldefs % nonzero if overriding \global specifications
% \escapechar % escape character in the out put of control sequence tokens
% \endlinechar % character placed at the right end of an input line
% \maxdeadcycles % upper bound on \deadcycles
% \fam % the current family number
% \mag % magnification ratio, times 1000

% dimen parameters (TeXbook page 374)
\maxdepth=4pt % maximum depth of boxes on main pages
\splitmaxdepth=\maxdimen % maximum depth of boxes on split pages
\boxmaxdepth=\maxdimen % maximum depth of boxes on explicit pages
\delimitershortfall=5pt % maximum space not covered by a delimiter
\nulldelimiterspace=1.2pt % width of a null delimiter
\scriptspace=0.5pt % extra space after a subscript or superscript
% \mathsurround % kerning before and after math in text
% \predisplaysize % length of text preceding a display
% \displaywidth % length of a line for displayed equation
% \displayindent % indentation of line for displayed equation

% glue parameters (TeXbook page 374)
\abovedisplayskip=12pt plus 3pt minus 9pt % extra glue just above displays
\abovedisplayshortskip=0pt plus 3pt % ditto, following short lines
\belowdisplayskip=12pt plus 3pt minus 9pt % extra glue just below displays
\belowdisplayshortskip=7pt plus 3pt minus 4pt % ditto, following short lines
\splittopskip=10pt % glue at top of split pages
% \tabskip % glue between aligned entries

% muglue parameters (TeXbook page 374)
\thinmuskip=3mu % thin space in math formulas
\medmuskip=4mu plus 2mu minus 4mu % medium space in math formulas
\thickmuskip=5mu plus 5mu % thick space in math formulas
\def\fmtname{dbp}
\xdef\fmtversion{\the\year-\the\month-\the\day-\the\time}
\dump
