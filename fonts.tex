\newcount\maxfontshrink
\newcount\maxfontstretch
\newcount\fontexpandstep

\def\fontfilename{\fontbasename\optionalwithprefix{-}{\fontshape}-\fontfigures\optionalwithprefix{-}{\fontvariant}-\fontencoding1\optionalwithprefix{ at }{\fontsize\relax}}
\def\fontshape{\optionalwithdefault{\fontweight\fontslant\fontopticalsize}{\fontdefaultshape}}
\def\fonttag{\fontstyle\expandafter\fontshape\fontfigures\fontvariant\fontencoding font}

% Declare a new font family.
% This specifies terms to identify parts of file names for this family
\def\newfamily#1#2#3#4#5#6{% #1 name for this family
  \expandafter\xdef\csname#1basename\endcsname{#2}% #2: the words that identify the family in this family's file names (e.g. GaramondPremrPro)
  \expandafter\xdef\csname#1regularweight\endcsname{#3}% #3: the possibly empty word that identifies the regular weight in this family's file names
  \expandafter\xdef\csname#1italicslant\endcsname{#4}% #4: the word that identifies italic slant in this family's file names
  \expandafter\xdef\csname#1boldweight\endcsname{#5}% #5: the word that identifies bold weight in this family's file names
  \expandafter\xdef\csname#1defaultshape\endcsname{#6}% #6: the possibly empty word that identifies regular roman book style in this family's file names
}

% Declare a new font style. This specifies the terms to use
% to build file names for this style.
\def\newstyle#1#2#3#4#5{% #1: a name for the style
  \expandafter\xdef\csname#1family\endcsname{#2}% #2: the name of this style's font family
  \expandafter\xdef\csname#1opticalsize\endcsname{#3}% #3: the word that identifies this style's optical size in the family's file names
  \expandafter\xdef\csname#1figures\endcsname{#4}% #4: the word that identifies this style's figure style in the family's file names
  \expandafter\xdef\csname#1size\endcsname{#5}% #5: the possibly empty point size at which to render this style
}

% Select a style and switch to its \rm font
\def\style#1{%
  \edef\fontstyle{#1}%
  \edef\fontfamily{\csname#1family\endcsname}%
  \edef\fontbasename{\csname\fontfamily basename\endcsname}%
  \edef\fontdefaultshape{\csname\fontfamily defaultshape\endcsname}%
  \edef\fontopticalsize{\csname#1opticalsize\endcsname}%
  \edef\fontfigures{\csname#1figures\endcsname}%
  \edef\fontsize{\csname#1size\endcsname}%
  \rm%
}

\def\withfontbasename#1{\edef\fontbasename{#1}}
\def\withfontencoding#1{\edef\fontencoding{#1}}
\def\withfontfigures#1{\edef\fontfigures{#1}}
\def\withfontopticalsize#1{\edef\fontopticalsize{#1}}
\def\withfontshape#1#2#3{\withfontweight{#1}\withfontslant{#2}\withfontopticalsize{#3}}
\def\withfontslant#1{\edef\fontslant{#1}}
\def\withfontvariant#1{\edef\fontvariant{#1}}
\def\withfontweight#1{\edef\fontweight{#1}}

\def\loadfont#1{%
  \ifcsname#1\endcsname\else%
    \global\expandafter\font\csname#1\endcsname=\fontfilename%
    \makeexpandable{\csname#1\endcsname}%
  \fi%
}

\def\selectfont{%
  \edef\selectedfonttag{\fonttag}%
  \loadfont{\selectedfonttag}%
  \csname\selectedfonttag\endcsname%
}

\def\rm{\withfontweight{\csname\fontfamily regularweight\endcsname}\withfontslant{}\withfontvariant{}\withfontencoding{t}\selectfont}
\def\it{\withfontslant{\csname\fontfamily italicslant\endcsname}\selectfont}
\def\bf{\withfontweight{\csname\fontfamily boldweight\endcsname}\selectfont}
\def\sc{\withfontvariant{sc}\selectfont}
\def\tl{\withfontvariant{titling}\withfontfigures{tlf}\selectfont}
\def\sym{\withfontencoding{ts}\selectfont}

\def\makeexpandable#1{\pdffontexpand#1 \maxfontstretch \maxfontshrink \fontexpandstep autoexpand}

\pdfadjustspacing=2

% TODO: Determine reasonable protrusions and turn this on
\pdfprotrudechars=0

\maxfontshrink=30
\maxfontstretch=30
\fontexpandstep=5
