% Cat codes for fundamental TeX features
\catcode`\\=0     % \ escape character
\catcode`\{=1     % { begin group
\catcode`\}=2     % } end group
\catcode`\$=3     % $ shift into and out of math mode
\catcode`\&=4     % & alignment tab
\catcode`\#=6     % # macro parameter prefix
\catcode`\^=7     % ^ superscript operator
\catcode`\_=8     % _ subscript operator
\catcode`\ =10    % space acts as space
\catcode`\~=13    % tilde is a control symbol
\catcode`\%=14    % % comment to end of line

\catcode`\^^M=5   % treat return as end of line
\catcode`\^^@=9   % char 0 is ignored
\catcode`\^^?=15  % char 255 is invalid TODO: Why?
\catcode`\^^I=10  % tab acts as space
\catcode`\^^L=13  % form feed is a control symbol

% Tab and return act as space
\def\^^I={\ } % Tab
\def\^^M={\ } % Return

% Form feed acts as \par
 \outer\def\^^L{\par}

% Tilde acts as nonbreaking space (aka "tie")
 \def~{\nbsp}

% TODO: What other basic syntax macros should I borrow from plain TeX?
\def\'#1{{\accent"13 #1}}
\chardef\$=`\$

% =0: assignments obey \global prefix
% <0: all assignments are local (\global prefix ignored)
% >0: all assignments are global (\global prefix implied)
\globaldefs=0

% \endlinechar % character placed at the right end of an input line
